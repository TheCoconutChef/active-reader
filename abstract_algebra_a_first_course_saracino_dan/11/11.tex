\documentclass[a4paper,10pt]{article}
\usepackage[utf8]{inputenc}
\usepackage[T1]{fontenc} % Use 8-bit encoding that has 256 glyphs
\usepackage{ae, aecompl}
\usepackage{amsmath,amsfonts,amsthm} % Math packages
\usepackage{enumitem}

\usepackage{fancyhdr} % Custom headers and footers

%opening
\title{11 Normal subgroups}

\begin{document}

\maketitle

\section*{11.5}
\begin{enumerate}
  \item $H$ un sous-groupe de $G$ 
  \item $K \triangleleft G$
\end{enumerate}
$\diamond$ $H \cap K \triangleleft H$
\\
\\
Car $H \triangleleft H$ et donc le résultat suit par \textbf{ex. 11.4}.

\section*{11.6}
\begin{enumerate}
	\item $H$ un sg de $G$
\end{enumerate}
$\diamond$ \textbf{$H \triangleleft G$ ssi $\forall x,y \in G$ on a $xy \in H \Leftrightarrow yx \in H$}
\\
\\
Soit $H \triangleleft G$ et soit de plus $x,y \in G$ tel que $xy \in H$. Alors $y \in x^{-1}H$. Or $x^{-1}H = Hx^{-1}$ (\textbf{thm. 11.1}). Donc $y \in Hx^{-1} \Leftrightarrow yx \in H$.
\\
\\
Soit alors $xy \in H$ ssi $yx \in H$. Alors $x^{-1}y \in H$ ssi $yx^{-1}H$, c'est-à-dire $y \in xH$ ssi $y \in Hx$. Donc $xH = Hx$ et donc $H \triangleleft G$ (\textbf{thm. 11.1}).

\section*{11.7}
\begin{enumerate}
	\item $H,K \triangleleft G$
	\item $H \cap K = \{e\}$
\end{enumerate}
$\diamond$ \textbf{$x \in H$ et $y \in K$ alors $xy = yx$}
\\
\\
Car $yxy^{-1} \in H$ par \textbf{thm. 11.1}. Aussi, $x^{-1} \in H$. Donc $yxy^{-1}x^{-1} \in H$. 
\\
\\
Mais de même, $xy^{-1}x^{-1} \in K$ et $y \in K$. Donc $yxy^{-1}x^{-1} \in K$. 
\\
\\
Donc $yxy^{-1}x^{-1} \in H \cap K$ et donc $yxy^{-1}x^{-1} = e \Rightarrow yx = xy$.

\section*{11.8}
\begin{enumerate}
	\item $N \triangleleft G$
	\item $H$ un sous-groupe de $G$
	\item $NH = \{nh : n \in N, h \in H\}$
\end{enumerate}
$\diamond$ \textbf{$NH$ est un sous-groupe de $G$}
\\
\\
Premièrement, que $NH = HN$. Car soit $nh \in NH$. Alors $nh \in Nh = hN$ (\textbf{hyp. 1}). Mais alors $nh \in HN$. Donc $NH \subseteq HN$ et de même dans l'autre direction.
\\
\\
Mais alors soit $nh \in NH$. Alors $h^{-1}n^{-1} = (nh)^{-1} \in HN = NH$. Donc tout élément de $NH$ possède son inverse dans $NH$. Aussi, $e \in NH$, car $N,H$ sont des groupes. Donc $NH$ est un sous-groupe de $G$. 

\section*{11.9}
\begin{enumerate}
	\item Mêmes hypothèses qu'en 11.8
	\item $H$ est normal
\end{enumerate}
$\diamond$ \textbf{$NH$ est normal}
\\
\\
Premièrement, que $gNHg^{-1} = (Ng)(g^{-}H)$. 
\\
\\
Car $gNHg^{-1} = \{gnhg^{-1} : gn \in gN, hg^{-1} \in Hg^{-1}\} =
 \{gnhg^{-1} : gn \in Ng, hg^{-1} \in g^{-1}H\} = (Ng)(g^{-1}H)$.
\\
\\
Or, $hn = (ng)(g^{-1}h) \in (Ng)(g^{-1}H)$. Donc $NH \subseteq gNHg^{-1}$. Donc $NH = gNHg^{-1}$ par \textbf{thm. 11.4} et donc $NH$ est normal par \textbf{thm 11.1}.
\end{document}

