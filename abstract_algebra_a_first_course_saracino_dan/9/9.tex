\documentclass[a4paper,10pt]{article}
\usepackage[utf8]{inputenc}
\usepackage[T1]{fontenc} % Use 8-bit encoding that has 256 glyphs
\usepackage{ae, aecompl}
\usepackage{amsmath,amsfonts,amsthm} % Math packages
\usepackage{enumitem}

\usepackage{fancyhdr} % Custom headers and footers

%opening
\title{9 Equivalence relation; Cosets}

\begin{document}

\maketitle

\section*{9.15}
\begin{enumerate}
  \item $G$ un groupe 
  \item $(a,b)R(c,d)$ ssi $ad = cb$
\end{enumerate}
$\diamond$ $\textbf{Trouver une condition sur G équivalente à } R \textbf{ est un relation d'équivalence}$
\\
\\
Si le groupe est abelien, on vérifie facilement qu'il s'agit d'une relation d'équivalence.
\\
\\
S'il s'agit d'une relation d'équivalence, alors les deux premières conditions sont satisfaites (réflexivité et symétrie).
\\
\\
On a alors
\begin{align*}
 & (a,a)R(e,e) \Leftrightarrow a = a \text{ et } (e,e)R(b,b) \Leftrightarrow b = b \\
 \Leftrightarrow \\
 & (a,a)R(b,b) \Leftrightarrow ab = ba
\end{align*}

\section*{9.16}
\begin{enumerate}
 \item $G$ un groupe
 \item $H$ un sous-groupe de $G$
 \item $K$ un sous-groupe de $H$
 \item $g_1 \dots g_n$ éléments de $G$ tels que les cosets droits de $H$ sont distincts
 \item $h_1 \dots h_m$ éléments de $H$ tels que les cosets droits de $K$ sont distincts
\end{enumerate}
$\diamond$ $ i \not= s \textbf{ ou } j \not= t \Rightarrow Kh_i g_j \not = Kh_s g_t$
\\
\\
On montre la converse. Soit $Kh_i g_j = Kh_s g_t$. Alors $h_i g_j g_t^{-1} h_s^{-1} \in K$.
\\
\\
Puisque $K$ est un sous-groupe de $H$, on a $h_i g_j g_t^{-1} h_s^{-1} \in H$. Donc $g_j g_t^{-1} \in H$ car 
$h_i^{-1}, h_s \in H$ un sous-groupe. Donc $H g_t = H g_j$ ($\textbf{Cor. 9.4}$). Ceci
implique $g_t = g_j$ ($\textbf{[4]}$).
\\
\\
Alors $Kh_i g_j = Kh_s g_t \Leftrightarrow Kh_i = Kh_s$ et donc $h_i = h_s$ ($\textbf{[5]}$).


\end{document}
