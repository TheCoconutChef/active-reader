\documentclass[a4paper,10pt]{article}
\usepackage[utf8]{inputenc}
\usepackage[T1]{fontenc} % Use 8-bit encoding that has 256 glyphs
\usepackage{ae, aecompl}
\usepackage{amsmath,amsfonts,amsthm} % Math packages
\usepackage{enumitem}
\usepackage[parfill]{parskip}
\usepackage{fancyhdr} % Custom headers and footers
\setlength{\parindent}{0cm}

%opening
\title{3.2 Fermat's little theorem}

\begin{document}

\maketitle

\begin{abstract}
Exercices de la section 3.2 Fermat's little theorem
\end{abstract}

\section*{1}

\begin{enumerate}[leftmargin=*,label={[\arabic*]}]
 \item $p$ premier
 \item $pgdc(p,n) = 1$
\end{enumerate}
$\diamond$ $p | n^{p-1}-1$
\\
\\
Car $p | n^p-n$ (\textbf{FLT}), donc $p | n(n^{p-1}-1$. Or, $pgdc(p,n)=1$ implique que $p \not | n$. 
Donc $p | n^{p-1}-1$ (\textbf{thm. 2.3}).

\section*{2}

\begin{enumerate}[leftmargin=*,label={[\arabic*]}]
 \item $pgdc(6,n)=1$
\end{enumerate}
$\diamond$ $6 | n^2 - 1$
\\
\\
Car de [1] ont déduit que 2 divise $(n-1)$ et $(n+1)$. De plus, 3 doit diviser $(n-1)$, $n$ ou $(n+1)$. 

Il suit de la que $2 \cdot 3 = 6 | (n-1)(n+1) = n^2 - 1$.

\section*{3}

$\diamond$ $n^5$ \textbf{ et } $n$ possèdent le même dernier chiffre

Car 5 divise $n^5 - n$ (\textbf{FLT}) et donc soit $n^5 - n$ finit par 0, soit il finit par 5. 

S'il finit par 0, alors c'est que les derniers chiffres sont égaux. 

Supposons alors qu'il finit par 5 et notons $b,d$ les derniers chiffre de $n^5,n$ respectivement.

Alors $|b-d| = 5$, et $b,d$ doivent nécessairement avoir parité différente. Or, $b$ est le dernier chiffre
de $d^5$, et doit donc avoir même parité, une contradiction. 

\end{document}
